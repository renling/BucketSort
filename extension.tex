
\section{Extensions}
\label{sec:extensions}

\subsection{Extension to Constant Client Storage}
\label{sec:O1client}
We now discuss how to extend our algorithms to the case where the client can only store $O(1)$ elements locally.

Each \textsc{MergeSplit} can be realized with a single invocation of bitonic sort.
Concretely, we first scan the two input buckets to count how many real elements should go to buckets $A'_0$ vs. $A'_1$, then tag the correct number of dummy elements going to either buckets, and finally perform a bitonic sort.

Next, we need to permute each output bucket obliviously with $O(1)$ local storage. 
This can be done as follows. 
First, assign each element in a bucket a uniformly random label of $\Theta(\log n)$ bits. 
Then, obliviously sort the elements by their random labels using bitonic sort. 
Since the labels are ``short'' (i.e., logarithmic in size), we may have collisions with $n^{-c}$ probability for some constant $c$, in which case we simply retry. 
In expectation, it succeeds in $1+o(1)$ trials. 

%or use the method in Chan et al.~\cite{opramdepth}[Lemma 10] \rl{what is this?}

Since we invoke $B/2$ instances of bitonic sort on $2Z$ elements at each level,
the runtime is roughly $\log B \cdot B/2 \cdot 2Z \log^2 (2Z)) \approx 2 n\log n \log^2 Z$. 
%In this case, our ORP and oblivious sort run in approximately $160n\log n$ time. \rl{Not competitive. is this part worth it?}

\subsection{Better Asymptotic Performance}
Our algorithms can also be extended to have better asymptotic performance.
For this instantiation, we use a primitive called oblivious tight compaction.
Oblivious tight compaction receives $n$ elements each marked as either 0 or 1, and outputs a permutation of the $n$ elements such that all elements marked 0 appear before the elements that are marked 1. 
It should not be hard to see that oblivious tight compaction can be used to achieve \textsc{MergeSplit}.
Using the $O(1)$-client-storage and $O(n)$-time oblivious tight compaction construction from~\cite{asharov2018optorama}, bucket oblivious sort achieves $O(n\log n + n\log^2Z)$ runtime and $O(1)$ client storage.
Setting $Z=\omega(1)\log n$, bucket oblivious sort achieves $O(n\log n)$ runtime, $O(1)$ client storage, and a negligible in $n$ error probability.

\subsection{Locality}
\label{sec:locality}
Algorithmic performance when the data is stored on disk has been studied in the external disk model (e.g.,~\cite{RuemmlerW94,ArgeFGV97,Vitter01,Vitter06}) and references within). Recently, Asharov et al.~\cite{AsharovCNPRS19} extended this study to oblivious algorithms. In this setting, an algorithm 
is said to have $(p, \ell)$ locality if it has access 
to $p$ disks and 
accesses in total $\ell$ discontiguous memory regions in all disks combined. As an example, it is not hard to see that merge sort is a non-oblivious sorting algorithm that sorts an array of size $n$ in $O(n \log n)$ and $(3,\log n)$-locality, whereas quick sort  is not local for any reasonable $p$. 
This locality metric is motivated by the fact that real-world storage
media such as disks support sequential accesses
much faster than random seeks. Thus an algorithm that 
makes mostly sequential accesses would execute much faster in practice than one that  
makes mostly random accesses --- even if the two have the same runtime in a standard
word-RAM model. 

Guided by this new metric, Asharov et al.~\cite{AsharovCNPRS19} consider how to design oblivious algorithms and ORAM schemes that achieve good locality. 
Since sorting is one of the most important
building blocks in the 
design of oblivious algorithms, 
inevitably Asharov et al.~\cite{AsharovCNPRS19}
show a locality-friendly sorting algorithm.
Concretely, they show that there is a specific way to implement
the bitonic sort meta-algorithm,
such that the entire algorithm requires accessing 
$O(\log^2 n)$ distinct memory regions (i.e., as many as the depth of the sorting network) 
require only 2 disks to be available --- in other words,
the algorithm achieves $(2, O(\log^2 n))$-locality.

We observe that our algorithm, when implemented properly, is a locality-friendly oblivious sorting algorithm. 
Our algorithm 
outperforms Asharov et al.~\cite{AsharovCNPRS19}'s  scheme 
by an almost logarithmic 
factor improvement in locality. % both runtime and locality.
To achieve this, the crux is to implement all $n/Z$ instances of 
$\textsc{MergeSplit}$ in the same layer of the butterfly network 
while accessing a small number of discontiguous regions. Specifically, the $\textsc{MergeSplit}$ operation works on 4 buckets at a time, while reading two buckets from the input layer, and writing to two consecutive buckets in the output layer. Moreover, the different invocations of $\textsc{MergeSplit}$ on the same layer deal with consecutive buckets. By carefully distributing the buckets among the different disks, and by using bitonic sort while implementing the $\textsc{MergeSplit}$ operation, we conclude:

\begin{corollary}
There exists a statistically oblivious sort algorithm which, except with $\approx e^{-Z/6}$ probability, completes in $O(n \log n \log^2 Z)$ work and with $(3, O(\log n \log^2 Z)$) locality.
\end{corollary}
%
%We observe that our algorithms, when implemented properly, have good data locality.
%The crux is to perform all $n/Z$ instances of 
%$\textsc{MergeSplit}$ in the same level of the butterfly network concurrently.
%We discuss this part in detail in Appendix~\ref{sec:locality}. 


%\medskip
%{\small
%\noindent
%{\bf Disclaimer.}
%This paper was prepared for information purposes jointly with the AI Research Group of JPMorgan Chase \& Co and its affiliates (“J.P.~Morgan”), and is not a product of the Research Department of J.P.~Morgan. J.P.~Morgan makes no explicit or implied representation and warranty and accepts no liability, for the completeness, accuracy or reliability of information, or the legal, compliance, financial, tax or accounting effects of matters contained herein. This document is not intended as investment research or investment advice, or a recommendation, offer or solicitation for the purchase or sale of any security, financial instrument, financial product or service, or to be used in any way for evaluating the merits of participating in any transaction.
%}

%%% Local Variables:
%%% mode: latex
%%% TeX-master: "main"
%%% End:
